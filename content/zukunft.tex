%%%%%%%%%%%%%%%%%%%%%%%%%%%%%%%%%%%%%%%%%%%%%%%%%%%%%%%%%%%%%%%%%%%%%%%%%%%%%%%
% Titel:   Design
% Autor:   Simon Plattner
% Datum:   28.04.2014
% Version: 1.0.0
%%%%%%%%%%%%%%%%%%%%%%%%%%%%%%%%%%%%%%%%%%%%%%%%%%%%%%%%%%%%%%%%%%%%%%%%%%%%%%%
%
%:::Change-Log:::
% Versionierung erfolgt auf folgende Gegebenheiten: -1. Release Versionen
%                                                   -2. Neue Kapitel
%                                                   -3. Fehlerkorrekturen
%
% 1.0.0       Erstellung der Datei
%%%%%%%%%%%%%%%%%%%%%%%%%%%%%%%%%%%%%%%%%%%%%%%%%%%%%%%%%%%%%%%%%%%%%%%%%%%%%%% 
\chapter{Fertigstellungsgrad}\label{ch:fertigstellungsgrad}
	%
	Das Projekt ist zum Zeitpunkt der Abgabe nicht vollst�ndig implementiert. Das liegt daran, dass aufgrund parallel laufender anderer Projekte die Zeit f�r weitere Arbeiten an diesem Projekt fehlte. Folgende Arbeiten sind nicht oder nur teilweise fertig gestellt:
	%
	\begin{itemize}
		\item{\textbf{"`Map"'-Klasse}: Diese Klasse ist zum jetzigen Zeitpunkt nur als "`Ger�st"' implementiert. Das bedeutet, die Klasse inklusive Attribute und Methoden ist vorhanden, die Methoden haben jedoch noch keine Funktion.}
		%
	\end{itemize}
	%
\chapter{Fortsetzung}\label{ch:fortsetzung}
	%
	Eine Weiterf�hrung des Projektes ist nicht vorgesehen und auch nicht empfehlenswert. Trotzdem lassen sich Use-Cases kreieren, in denen eine Weiterf�hrung des Projekts w�nschenswert w�re. Als m�gliche Erweiterung k�nnten beispielsweise folgende Funktionen umgesetzt werden:
	%
	\begin{itemize}
		\item{\textbf{Spiel pausieren}: Eine Funktion zum vor�bergehenden Unterbrechen eines laufende Spiels k�nnte umgesetzt werden}
		%
		\item{\textbf{Punktestand festhalten}: F�r das laufende Spiel k�nnten Punkte gesammelt werden. Diese Funktion w�re insbesondere in Kombination mit einer Bestenliste vorstellbar}
		%
		\item{\textbf{Mehrspielermodus}: Mit grossem Aufwand k�nnte das Spiel f�r einen Mehrspielerbetrieb umger�stet werden} 
	\end{itemize}
	%
	\section{Bekannte Bugs}
		%
		Die nachfolgend genannten Bugs sind bekannt:
		%
		\begin{itemize}
			\item{Das Tetromino kann sich aus der Map bewegen und bleibt nicht unten liegen. Dieser Bug ist auf die unvollst�ndige Implementation der Klasse "`Map"' zur�ckzuf�hren. (Siehe \autoref{ch:fertigstellungsgrad})}
			%
			\item{Mit der Richtungstaste \framebox{$\uparrow$}} wird das Tetromino entfernt und ein neues am Startpunkt erstellt. Dieses Verhalten wurde absichtlich implementiert, damit die einzelnen Tetrominoklassen zum jetzigen Softwarestand getestet werden k�nnen.
			%
		\end{itemize}