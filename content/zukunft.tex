%%%%%%%%%%%%%%%%%%%%%%%%%%%%%%%%%%%%%%%%%%%%%%%%%%%%%%%%%%%%%%%%%%%%%%%%%%%%%%%
% Titel:   Design
% Autor:   Simon Plattner
% Datum:   28.04.2014
% Version: 1.0.0
%%%%%%%%%%%%%%%%%%%%%%%%%%%%%%%%%%%%%%%%%%%%%%%%%%%%%%%%%%%%%%%%%%%%%%%%%%%%%%%
%
%:::Change-Log:::
% Versionierung erfolgt auf folgende Gegebenheiten: -1. Release Versionen
%                                                   -2. Neue Kapitel
%                                                   -3. Fehlerkorrekturen
%
% 1.0.0       Erstellung der Datei
%%%%%%%%%%%%%%%%%%%%%%%%%%%%%%%%%%%%%%%%%%%%%%%%%%%%%%%%%%%%%%%%%%%%%%%%%%%%%%%    
\chapter{Fortsetzung}\label{ch:fortsetzung}
	%
	Eine Weiterf�hrung des Projektes ist nicht vorgesehen. Trotzdem lassen sich Use-Cases kreieren, in denen eine Weiterf�hrung des Projekts w�nschenswert w�re. Als m�gliche Erweiterung k�nnten beispielsweise folgende Funktionen umgesetzt werden:
	%
	\begin{itemize}
		\item{\textbf{Spiel pausieren}: Eine Funktion zum vor�bergehenden Unterbrechen eines laufende Spiels k�nnte umgesetzt werden}
		%
		\item{\textbf{Punktestand festhalten}: F�r das laufende Spiel k�nnten Punkte gesammelt werden. Diese Funktion w�re insbesondere in Kombination mit einer Bestenliste vorstellbar}
		%
		\item{\textbf{Mehrspielermodus}: Mit grossem Aufwand k�nnte das Spiel f�r einen Mehrspielerbetrieb umger�stet werden} 
	\end{itemize}
	%
	\section{Bekannte Bugs}
		%
		Die nachfolgend genannten Bugs sind bekannt:
		%
		\todo{fix the list of bugs}
		%
		\begin{itemize}
			\item{Propulsion is a lie}
			%
			\item{This statement is true}
			%
			\item{At the end, there will be a overwhelming number of playing tables... but no second earlier} 
		\end{itemize}