%%%%%%%%%%%%%%%%%%%%%%%%%%%%%%%%%%%%%%%%%%%%%%%%%%%%%%%%%%%%%%%%%%%%%%%%%%%%%%%
% Titel:   Beispiele
% Autor:   Simon Plattner
% Datum:   24.04.2014
% Version: 1.0.0
%%%%%%%%%%%%%%%%%%%%%%%%%%%%%%%%%%%%%%%%%%%%%%%%%%%%%%%%%%%%%%%%%%%%%%%%%%%%%%%
%
%:::Change-Log:::
% Versionierung erfolgt auf folgende Gegebenheiten: -1. Release Versionen
%                                                   -2. Neue Kapitel
%                                                   -3. Fehlerkorrekturen
%
% 1.0.0       Erstellung der Datei
%%%%%%%%%%%%%%%%%%%%%%%%%%%%%%%%%%%%%%%%%%%%%%%%%%%%%%%%%%%%%%%%%%%%%%%%%%%%%%%    
\chapter{Analyse}\label{ch:analyse}
	%
	Zuerst soll eine Analyse des Projektes durchgef�hrt werden. Dazu werden verschiedene UML Diagramme erzeugt. Als erstes kommt das Use-Case-Diagramm zum Einsatz. Abbildung \ref{abb:use_case} zeigt einige solche Anwendungsf�lle auf.
	%
	\image{content/UML/UseCase}{width=0.8\textwidth}[Anwendungsfall-Diagramm][abb:use_case]
	%
	\section{Use-Cases}
		%
		Im Nachfolgenden sollen einige Anwendungsf�lle betrachtet werden. Anwendungsfall \textit{Spielstein drehen} (in Abschnitt \ref{ss:use_case_spielstein_drehen}) ist als einziger voll umf�nglich beschrieben. \par
		%
		\subsection{Spiel starten}
			%
			Tabelle \ref{tab:use_case_spiel_starten} zeigt den Anwendungsfall, bei dem ein neues Spiel gestartet wird. Dieser Fall kommt genau einmal beim Start der Applikation vor. Das Spiel kann nicht neu gestartet, ohne das Programm neu zu starten. Ein erneutes Auftreten des Anwendungsfalls ist damit weg bedingt.
			%
			\begin{table}[htbp]
				\centering
				\begin{tabular}{ p{1.5cm} p{2cm} p{8cm} } 
					\hline
					Fall Nr. & 1 & Spiel starten \\
					\hline
					\multicolumn{2}{ l }{Beschreibung:} & Der Spieler startet ein neues Spiel. Das Spielfeld wird geleert, der GameHandler wird gestartet. \\  
					\hline
					\multicolumn{2}{ l }{Vorbedingung:} & Das System ist eingeschaltet. Es l�uft noch kein Spiel. \\
					\hline
				\end{tabular}
				\caption{Anwendungsfall \textit{Spiel starten}}
				\label{tab:use_case_spiel_starten}  
			\end{table}
			%
		\subsection{Spielfeld aktualisieren}
			%
			Tabelle \ref{tab:use_case_spielfeld_aktualisieren} zeigt den Anwendungsfall, bei dem das Spielfeld durch den GameHandler aktualisiert wird. Dieser Fall tritt periodisch auf. F�r den Aufruf entsprechender Methoden sorgt ein Softwaretimer.
			%
			\begin{table}[htbp]
				\centering
				\begin{tabular}{ p{1.5cm} p{2cm} p{8cm} } 
					\hline
					Fall Nr. & 2 & Spielfeld aktualisieren \\
					\hline
					\multicolumn{2}{ l }{Beschreibung:} & Der GameHandler reagiert auf Benutzereingaben und aktualisiert das Spielfeld. Wenn eine Reihe komplett wurde, wird diese entfernt. Wenn das Spielfeld voll ist, wird das Spiel abgebrochen. \\  
					\hline
					\multicolumn{2}{ l }{Vorbedingung:} & Das Spiel l�uft. Der Player macht eine Bewegung oder eine Drehung des Spielsteins oder der GameHandler aktualisiert das Spiel. \\
					\hline
				\end{tabular}
				\caption{Anwendungsfall \textit{Spielfeld aktualisieren}}
				\label{tab:use_case_spielfeld_aktualisieren}  
			\end{table}
			%
			
			
			
			
			
			
			
			
			
			
			
			
			
			
			
			
		\subsection{Spielstein drehen}\label{ss:use_case_spielstein_drehen}
			%
			Tabelle \ref{tab:use_case_spielstein_drehen} zeigt den Anwendungsfall, bei dem ein Spielstein (ein sogenanntes \textit{Tetromino}\footnote{Die Bezeichnung \textit{Tetromino} steht im englischen f�r eine Geometrische Figur, die aus vier Quadraten aufgebaut ist (und einigen weiteren Anforderungen gen�gt). Die klassischen Spielsteine aus \textit{TETRIS} sind Tetrominos.}) gedreht wird. Das passiert unabh�ngig von den Zeitpunkten, zu denen der GameHandler das Spielfeld aktualisieren l�sst.\todo{in tabelle \ref{tab:use_case_spielstein_drehen} referenz auf sequenzdiagram einbetten}
			%
			\begin{table}[htbp]
				\centering
				\begin{tabular}{ p{1.5cm} p{2cm} p{8cm} } 
					\hline
					Fall Nr. & 3 & Spielstein drehen \\
					\hline
					\multicolumn{2}{ l }{Beschreibung:} & Der GameHandler reagiert auf Benutzereingaben und aktualisiert das Spielfeld. Wenn eine Reihe komplett wurde, wird diese entfernt. Wenn das Spielfeld voll ist, wird das Spiel abgebrochen. \\  
					\hline
					\multicolumn{2}{ l }{Vorbedingung:} & Das Spiel l�uft. \\
					\hline
					%
					%
					\multicolumn{2}{ l }{Ablauf:} & \\
					\multicolumn{2}{ c }{E1)} & asdf \\
					\multicolumn{2}{ c }{A1)} & bsdf \\
					\multicolumn{2}{ c }{E2)} & asdf \\
					\multicolumn{2}{ c }{A2)} & bsdf \\
					%
					%
					\hline
					\multicolumn{2}{ l }{Auswirkungen:} & Das Tetromino hat eine neue Ausrichtung. \\
					\hline
					\multicolumn{2}{ l }{Weitere Informationen:} & Drehen ist nur m�glich, falls dabei eine zul�ssige Bewegung statt finden (beispielsweise darf das Tetromino nicht in ein anderes Tetromino \enquote{hineingedreht} werden). \\
					\hline
					\multicolumn{2}{ l }{Diagramme:} & TODOTODOTODOTODO \\
					\hline
				\end{tabular}
				\caption{Anwendungsfall \textit{Spielstein drehen}}
				\label{tab:use_case_spielstein_drehen}  
			\end{table}
			%

	
	
	
	
	
	
	
	
	
	
%	
%	\begin{table}[htbp]
%		\centering
%		\begin{tabular}{|l|l|l|} 
%			\hline
%			\rowcolor{bfhblue}
%			\textcolor{white}{Spalte 1} & \textcolor{white}{Spalte 2} & \textcolor{white}{Spalte 3}\\
%			\hline
%			Ich & bin & da \\
%			\hline
%			\multicolumn{2}{|l|}{Zwei Spalten vereinen} & Das geht!\\
%			\hline
%		\end{tabular}
%		\caption{Messmittelliste der Messung Messleitung}
%		\label{tab:tabelle}  
%	\end{table}
	
	
	
	
	
	
	
	
	
	
	
	
	
	
	
	
	
	
	
	
	
	
	
	
	
	