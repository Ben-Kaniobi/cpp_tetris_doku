%%%%%%%%%%%%%%%%%%%%%%%%%%%%%%%%%%%%%%%%%%%%%%%%%%%%%%%%%%%%%%%%%%%%%%%%%%%%%%%
% Titel:   Abstract
% Autor:   S. Grossenbacher
% Datum:   27.09.2013
% Version: 1.0.0
%%%%%%%%%%%%%%%%%%%%%%%%%%%%%%%%%%%%%%%%%%%%%%%%%%%%%%%%%%%%%%%%%%%%%%%%%%%%%%%

%:::Change-Log:::
% Versionierung erfolgt auf folgende Gegebenheiten: -1. Stelle Semester
%                                                   -2. Stelle neuer Inhalt
%                                                   -3. Fehlerkorrekturen
%
% 1.0.0       Erstellung der Datei

%%%%%%%%%%%%%%%%%%%%%%%%%%%%%%%%%%%%%%%%%%%%%%%%%%%%%%%%%%%%%%%%%%%%%%%%%%%%%%%
\chapter*{Abstract}
Im \cpp\ Unterricht der Berner Fachhochschule wird im Rahmen eines Projektes ein \enquote{Klon} des bekannten Spiels \textit{TETRIS}\footnote{Wir haben keinerlei Rechte weder am urspr�nglichen Spiel noch am Namen desselben. Da das Wort \textit{TETRIS} jedoch bereits im normalen Sprachgebrauch �blich ist, verwenden wir in diesem Dokument die Bezeichnung \textit{TETRIS} f�r die Spielidee, nicht jedoch als Produktname.}(welches dieses Jahr das 30-j�hrige Jubil�um feiert) implementiert. Das Projekt ist weitestgehend objektorientiert aufgebaut und umgesetzt. Die Entwicklung fand dabei haupts�chlich im Bereich der Spiellogik statt. Die Darstellung wird durch nicht zum Projekt geh�rende Komponenten erledigt.\par
%
Das Projekt dient dazu, das gelernte Wissen aus dem zugeh�rigen Vorlesungsmodul anzuwenden und zu verinnerlichen. Die Dokumentation dient dazu, die gewonnenen Kenntnisse im Bereich UML zu demonstrieren.\par
%
Aufgrund anderer, parallel verlaufener Projekte mit h�herer Priorit�t, konnte das Spiel nicht vollst�ndig fertig gestellt werden. Da das Design bereits f�r das gesamte Spiel vorliegt, musste jedoch ausschliesslich auf \textit{die Implementierung} einiger Elemente verzichtet werden.\par
%
Eine Weiterentwicklung an diesem Projekt scheint wenig Sinnvoll. Es ist von Anfang an nur dazu gedacht, das erworbene Wissen in der Praxis anzuwenden und ist daher eher als \cpp\ Sandbox der Autoren aufzufassen und sollte nicht weiterverwendet werden.

%\chapter*{Vorwort}
%Einmal mehr im Werdegang eines angehenden Ingenieurs steht die Umsetzung eines Projektes zwischen uns und einer guten Note. Einmal mehr macht diese gute Note nur einen kleinen Teil der gesamten Bewertung des Moduls aus. Und ebenfalls einmal mehr macht die Dokumentation nur einen kleinen Teil der Bewertung des Projektes aus. Aus diesem Grund soll diese Dokumentation unter anderem dazu dienen, dass wir uns im Kurzfassen �ben k�nnen. Es soll auf abrundendes Prosa verzichtet werden, wo immer dadurch sinnvoll Zeit gespart werden kann.\par
%%
%Diese Dokumentation richtet sich demnach an den bewertenden Dozenten, welcher sowohl die Entwicklungsschritte und die eingesetzten Methoden als auch die Sprache an sich sehr gut beherrscht. Sie soll aufzeigen, was wir uns �berlegt haben, jedoch auf alles allgemeine oder unn�tige verzichten.\par
%%
%Um den Projekt-Overhead durch die Dokumentation m�glichst klein zu halten, verzichten wir auf das Anf�gen von Stichwortverzeichnissen, ausf�hrlichen Einleitungen, �berm�ssiger Bebilderung und so �hnlichem.\par
%%
%Trotzdem w�nschen wir dem Leser viel Vergn�gen!
